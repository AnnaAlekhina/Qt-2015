\documentclass[12pt]{report}
\usepackage[utf8]{inputenc}
\usepackage[T2A]{fontenc}
\usepackage[russian]{babel}
\usepackage{amsmath}
\usepackage[backend=biber,style=authoryear,autocite=footnote]{biblatex}
\usepackage{textcase}
\usepackage[left=2.5cm,right=2.5cm,top=2.5cm,bottom=2.5cm]{geometry}
\usepackage{listings}

\usepackage{xcolor}
\usepackage{color}
\usepackage[autostyle]{csquotes}
\addbibresource{biblio.bib}
\begin{document}	
\title{Отчет}
\author{Алехин Тарас\\ 
	Гаус Анна}
\begin{titlepage}
\begin{center}
	Федеральное государственное автономное образовательное учреждение\\
	высшего профессионального образования\\
	\MakeUppercase{<<ЮЖНЫЙ ФЕДЕРАЛЬНЫЙ УНИВЕРСИТЕТ>>}\\[1cm]
	Факультет математики, механики и компьютерных наук\\[1cm]
	\vspace*{\fill}
	{\LARGE Алехин Т.С. Гаус А.В.\\Отчет\\по летней практике}
	\vspace*{\fill}
\begin{flushright}
	Преподаватель по летней практике:\\ Лошкарев И.В.
\end{flushright}
	\vspace*{\fill}
	Ростов-на-Дону\\
	2015
\end{center}
\end{titlepage}
\vfill\tableofcontents\clearpage

\begin{abstract}
Это отчет по летней практике 2015 года по теме 
"Очередь с приоритетом на базе односвязного списка"

Очередь с приоритетом  –  структура данных с доступом к элементам «первый пришёл – первый вышел». Добавление элемента возможно лишь в конец очереди, выборка – только из начала очереди, при этом выбранный элемент из очереди удаляется.
\end{abstract}
\chapter{Проект}
\section{Постановка задачи}

Очередь с приоритетом на базе односвязного списка

Очередь с приоритетом \autocite{wikiList}  –  структура данных с доступом к элементам «первый пришёл – первый вышел». Добавление элемента возможно лишь в конец очереди, выборка – только из начала очереди, при этом выбранный элемент из очереди удаляется.\\


Реализовать класс:

Класс должен содержать методы:

добавление элемента в очередь, с заданным приоритета;

получение значения очередного элемента;

извлечение очередного элемента;

проверка очереди на пустоту;\\


Для класса необходимо перегрузить следующие операторы:

operator>> извлечение элемента из очереди;

operator<< добавление элемента в очередь;

operator+ слияние двух очередей;

ввод/вывод в поток.\\

Используя очередь строк решить задачу:


Дан файл строк. Выдать строки в порядке убывания их длин. Используя один проход по файлу.

\section{Метод Решения}
Для решения задачи использовалась программа QT\\
Был создан совместный репозиторий на gitHub\\
Класс максимально покрыт тестами\\ 
	
\section{Реализация}
	\lstdefinestyle{customc}{
		belowcaptionskip=1\baselineskip,
		breaklines=true,
		frame=L,
		xleftmargin=\parindent,
		language=C++,
		showstringspaces=false,
		basicstyle=\footnotesize\ttfamily,
		keywordstyle=\bfseries\color{green!40!black},
		commentstyle=\itshape\color{purple!40!black},
		identifierstyle=\color{blue},
		stringstyle=\color{orange}
	}
\lstset{language=C++,style=customc}  
\begin{lstlisting}
	#ifndef QUEUE_H
	#define QUEUE_H
	
	#include <QObject>
	#include "mylist.h"
	#include <iostream>
	using namespace std;
	
	class queue : public MyList
	{
	Q_OBJECT
	
	public:
	explicit queue(QObject *parent = 0);
	queue(queue const &A);
	queue &operator=(queue &A);
	void addPrior(MyListData x);
	MyListData getMostPr();
	MyListData remMostPr();
	bool is_empty();  
	queue& operator>>(MyListData &x);
	queue& operator<<(MyListData const &x); 
	queue operator+(queue &A);
	
	friend ostream& operator<<(ostream & os,  queue const & d);
	friend istream& operator>>(istream & is, queue & d);
	};
	ostream& operator<<(ostream & os,  queue const & d);
	istream& operator>>(istream & is, queue & d);
	
	#endif 
\end{lstlisting}

\lstdefinestyle{customc}{
		belowcaptionskip=1\baselineskip,
		breaklines=true,
		frame=L,
		xleftmargin=\parindent,
		language=C++,
		showstringspaces=false,
		basicstyle=\footnotesize\ttfamily,
		keywordstyle=\bfseries\color{green!40!black},
		commentstyle=\itshape\color{purple!40!black},
		identifierstyle=\color{blue},
		stringstyle=\color{orange}
	}
\lstset{language=C++,style=customc}          % Задаем язык исходного кода
\begin{lstlisting}                
	#include <QCoreApplication>
	#include <iostream>
	#include <fstream>
	#include "test_mylist.h"
	#include "test_queue.h"
	#include "mylist.h"
	#include "queue.h"
	using namespace std;
	
	int main(int argc, char *argv[])
	{
		QCoreApplication a(argc, argv);
		QTest::qExec(new test_queue, argc, argv);	
		queue que;
		char ch[30];
		cout<<"name file: "<<endl;
		cin.getline(ch,30);
		MyListData dat;
		ifstream fin(ch);
		ofstream fout("NewFile.txt");
		cout<<"create new file "<<endl;
		if(fin.is_open()){
				while(!fin.eof()){
				fin.getline(dat.str,30);
				dat.priority=strlen(dat.str);
				que.addPrior(dat);
			}
		}
		fout<<que;
		fout.close();
		fin.close();
		return 0;
	}
\end{lstlisting}                  

\printbibliography
\end{document}

